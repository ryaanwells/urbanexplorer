\chapter{Results and Analysis}\label{ch_results}

\section{Evaluation of Interface and Workflow}
\subsection{Initial Design}

\subsection{Final Design}

\section{Evaluation of Gamification}

\section{Evaluation of Platform}
PhoneGap has its merits as a mobile platform: you can achieve rapid
development cycles by taking advantage of strong JavaScript and CSS
frameworks, a ``write once, deploy anywhere'' strategy for multiple
platform development, and good integration with the phones native
functionality. However it is the details of the integration with the
native functionality that can leave wanting, especially for
geolocation information.

As is discussed in Section \ref{sec:location_mgmt}, we use the PhoneGap
provided location watch service to receive location information. It
was discovered during development that if the PhoneGap provided
location watch service could not obtain location information for any
reason then the request would fail silently. Geolocation information
cannot always be obtained due to factors outwith our control and so
failing silently does not allow the developer to enforce any form of
persistence when searching for location information. In the case that
we could not obtain location information it may be desirable to
immediately restart searching for location information to provide the
user with better quality of service, however other factors such as
power consumption then come in to effect which may be more detrimental
than a few missed waypoints. This would need to be further analysed in
future work to properly justify what the correct course of behaviour
is in this situation.

The biggest challenge with the PhoneGap platform was the inconsistency
within the documentation of the platform and also the instability of
the upgrade process. 

The PhoneGap platform is built of two main modules: the PhoneGap
module which is concerned with the developers interface to the phone,
and the Cordova module which is concerned with the device management.
There is disparity between platform installation documentation and the
plugin documentation (for the geolocation module
etc)\cite{phonegap_install, phonegap_cli,
  phonegap_geolocationAccessingFeature}. This inconsistency does not
fill the developer with confidence when the official recommendations
cannot agree on correct procedures. 

Considerable time was also lost during the first round of user
evaluation due to an undocumented step required during platform
updating. When an Android project is created a JAR is installed that
is responsible for interfacing with the device but updating the
PhoneGap project did not update this JAR. Since the JAR was not
updated it did not have new functionality that the project was
attempting to access and so was crashing the app. A solution was
eventually discovered on an unrelated thread\cite{pluginFix} and user
evaluation was able to continue. Time was unnecessarily lost over bad
documentation and wrong assumptions by the platform developers.

Other small issues like a provided build script to build a ``release''
version of the app for the Android Play Store builds a ``debug''
version instead, each time you build the app the version number and
release number are reset to their default values, an XML configuration
file for the app is ignored when searching for the main ``index.html''
file (root of the application), amongst others.

It is for these reasons that I could not recommend PhoneGap as a
platform that is viable for a production quality app. Until the
documentation and platform become more reliable and consistent the
benefits of cross platform development do not outweight the loss of
fidelity incurred by non native implementation.
