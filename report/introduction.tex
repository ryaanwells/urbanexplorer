\chapter{Introduction}\label{ch_intro}

\section{Motivation / Context}
The number of smartphones being sold per year is
increasing\cite{phones_gartner, phones_guardian} and nearly
three-quarters of exercisers use technology to support their
workouts in some way\cite{lifefitness}. This project aims to utilise
this mass market 
acceptance of technology in exercise and combine it with game like
elements to evaluate the effects of games and
exercise. Specifically, this project will look at outdoor
cardiovascular exercise where the user is required to exercise with
their phone with them.Studies have shown that physical exercise
between 1 hour a day for children over five years of age and as
little as 2.5 hours a week for adults has a huge positive impact on
our physical and emotional wellbeing\cite{govsurvey, amsurvey} .

The 2006 Health Survey for England\cite{exercise} states that: 
\begin{quote}
  Physical inactivity is associated with all-cause mortality and
  many chronic diseases, including ischaemic heart disease, diabetes,
  certain cancers, and obesity \dots \ Many people attribute their
  failure to achieve the target recommendations to a lack of time to
  take exercise. 
\end{quote} 
We intend to approach this problem by targeting the perception that
there is a ``lack of time to take exercise''. 

A later Health Survey for England in 2012 noted that the change in
definition as to what the exercise guidelines were implied that more
people than before were meeting the exercise guidelines
\cite{exercise_2012}. These guidelines allow for exercise sessions of
10 minutes or more where previously exercise only counted if it was
more than 30 minutes in length. Automatically this will include
more people than the previous survey as the criteria is less
restrictive, however we can use this to our advantage..

Since the guidelines now allow exercise sessions that can be as little
as 10 minutes we can use this to deal with the ``lack of time to take
exercise'' excuse and encourage users to use these small time
intervals to undertake exercise. 

\section{Research Questions}

The application of game like elements to exercise is nothing new. As
will be
discussed in Section \ref{sec_comparison}, other applications have
taken different approaches to incentivising exercise. To the best of
my knowledge no other product uses exactly the same game experience as
the one I propose in this dissertation, so the main points for
analysis are with regards to the distinctive features of this
experience. 

This dissertation aims to analyse the main two following points within
the scope of the project:

\begin{enumerate}
  \item How valid is the application of badge-based gamification to
    the context of outdoor exercise? I intend to gauge this through
    measuring the distance travelled and length of exercise duration
    of users and determine if a change occurs.
  \item How valid is the platform for this kind of encouragement?
    With user feedback as a basis, I will propose whether or not
    mobile phones are a suitable platform based on preferences of
    users and ease of use.
\end{enumerate}


\section{Targeted users}
\label{sec:targeted_users}
Consider an individual who has a relatively short distance to travel,
such as a commute to work,
and who doesn't exercise regularly. This individual could either
travel this distance on their own accord or use a service such as a
car or take public transport. With no external encouragement, and low
self control, the individual may be likely to take the car or public
transport and lose out on all benefits of exercise. 

By gamifying this travel time we hope that the user decides not to
take public transport or travel by car so that they can achieve the
goals of the game. This positive encouragement will help the user to
begin exercising and be rewarded for doing so. Gradually over time as
users start to engage with the game more often we hope that users
start to willingly extend their commute, colloquially ``taking the
long route'', or simply exercise more often just to ensure that they
receive an achievement. If this happens then the game has been
successful as through using this application the user has exercised
more than they would have without it. The short term rewards will
initially help these users the most: since they do not exercise
often they may need quicker encouragement to help push them to
exercise more regularly.

Also consider a user who frequently exercises and repeatedly travels
the same route, or the same collection of routes when exercising. This
individual may easily tire of the journey or bad weather may
discourage the individual just enough that they decide to move their
exercise to a controlled environment such as a gym where the benefits
are not as profound, or give up entirely. 

If we can encourage this user to continue exercising by giving the
user a reward structure they wish to invest in then we can help this
user continue to benefit from exercise. In this case the game is a
success if it diverts the user from changing away from their current
exercise regime. Long term rewards will benefit these users as they
can use these to help encourage themselves to maintain the current
rate that they are exercising at. These will be reinforced with the
rewards from the short term goals in a similar way to the previous
example. 

\section{Summary of Contributions}

This report presents the following contributions: 

\begin{itemize}
  \item A brief overview of gamification;
  \item An analysis of the current state of the market for exercise
    apps and devices;
  \item An exploration of the requirements, design decisions and
    implementation of ``Urban Explorer'' - the application built for
    the analysis of this type of gamification;
  \item The ``Urban Explorer'' application, which can also be found on
    the Google Play Store \cite{app_store_link};
  \item The results from user evaluation and testing.
\end{itemize}

\section{Outline of the Report}

We will start by examining the current state of the market in Chapter
\ref{ch_background} by
discussing other implementations of gamification in mobile
apps. Through this examination, we will discover the benefits and
pitfalls of each implementation and use this to form the basis of our
implementation. 

We define the overall goals of the project in Chapter \ref{ch_method},
specifically how we will use gamification to deliver our
implementation and what we hope to achieve in the scope of this
project.

An analysis of our implementation will then be presented in Chapter
\ref{ch_results}. This is based on user testing and experience, and evaluation of the
overall suitability of the platform.
