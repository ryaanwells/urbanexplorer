\chapter{Introduction}\label{ch_intro}

\section{Motivation / Context}
The number of smartphones being sold per year is
increasing\cite{phones_gartner, phones_guardian} and nearly
three-quarters of exercisers use technology to support their
workouts in some way\cite{lifefitness}. This project aims to utilize this mass market
acceptance of technology in exercise and combine it with game like
elements to begin an evaluation on the effects of games and
exercise. Specifically, this project will look at outdoor
cardiovascular exercise where the user is required to exercise with
their phone with them.

The main problem I aim to address is the boredom that can arise
through exercise or the reluctance to do so in the first
place. 

Consider a frequent exerciser who repeatedly travels the same route,
or the same collection of routes when exercising. This individual may
easily tire of the journey or bad weather may discourage the
individual just enough that they decide to move their exercise to a
controlled environment such as a gym where the benefits are not as profound.

Also consider a different individual who has a relatively short distance to
travel and who doesn't exercise regularly. This individual could
either travel this distance on their own accord or use a service such
as a car or take public transport. With no outside encouragement, and
low self control, the individual may be likely use the service based
travel method and lose out on all benefits of exercise.

Both individuals above would benefit from positive encouragement for
exercise. I intend to provide this positive encouragement through
bringing game like elements to the exercise routine in such a way that
it can cater for heavy and casual users alike.

The project will allow users to travel along a ``virtual route''
while they traverse the physical route they are exercising on and will
gain achievements for completing these ``virtual routes''.

\section{Research Questions}

The application of game like elements to exercise is nothing new. As
discussed in Section \ref{sec_comparison}, other applications have
taken different approaches to incentivising exercise. To the best of
my knowledge no other product uses exactly the same game experience as
the one I propose in this dissertation, so the main points for
analysis are with regards to the distinctive features of this
experience. 

This dissertation aims to analyse the main two following points within
the scope of the project:

\begin{enumerate}
  \item How valid is the application of badge-based gamification to
    the context of outdoor exercise? I intend to gauge this through
    measuring the distance travelled and length of exercise duration
    of users and determine if a change occurs.
  \item How valid is the platform for this kind of encouragement?
    With user feedback as a basis, I will propose whether or not
    mobile phones are a suitable platform based on preferences of
    users and ease of use.
\end{enumerate}

\section{Summary of the Contributions}

This project required producing a mobile app to provide the
gamification and a webserver to manage data provision and result
gathering. 

The mobile app takes the form of an Android App which is hosted on the
Google Play Store \cite{app_store_link}. It is recommended for
convenience to download the app directly from here, however you can
install from source or the generated apk if desired. The version that
is hosted on the app store will not change between the time of
submission and results being published.

The webserver was developed in a RESTful style using the django
middleware framework\cite{django} and tastypie, a django module for
generating a REST style API\cite{tastypie}. Some business logic does
not fit with the REST methodology, such as registration and exercise
session management, so specific views have been implemented to cater
for this. These are constructed using solely the django framework.

When in use, the mobile app requires an internet connection to obtain
data from the webserver and to communicate location information to the
webserver. Since this is an application intended for use when the user
is not on a home network, care has been taken to minimise the network
traffic required. Content is managed in the app by utilising explicit
caching: this retains local copies of data objects that will not
change during the runtime of the app. 

Care was also taken when designing the interaction for managing an
exercise session, with the overall goal of further reducing the
network cost. Inspiration was taken from some of the REST principles,
but this work flow does not adhere to all of these principles. 

These are explained in full in Section \ref{sec_notable}.

\section{The Structure or Outline of the Report}

We will start by examining the current state of the market by
discussing other implementations of gamification in mobile
apps. Through this examination, we will discover the benefits and
pitfalls of each implementation and use this to form the basis of our
implementation. 

We will then define the overall goals of the project: how we will use
gamification to deliver our implementation and what we hope to
achieve in the scope of this project. 

An analysis of our implementation will then be presented. This is
based on user testing and evaluation of the overall suitability of
the platform and experience.
