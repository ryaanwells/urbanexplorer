% This example An LaTeX document showing how to use the l3proj class to
% write your report. Use pdflatex and bibtex to process the file, creating 
% a PDF file as output (there is no need to use dvips when using pdflatex).

% Modified 

\documentclass{l4proj}
\usepackage{url}
\usepackage{color}
\usepackage{graphics,graphicx}
\usepackage{epsfig}
\usepackage{epstopdf}
\usepackage{colortbl}
\usepackage{multirow}
\usepackage{booktabs}
\usepackage{ifthen}
\usepackage{float}
\usepackage[superscript,biblabel]{cite}
\usepackage{hyperref}
\usepackage{mathtools}

\begin{document}

\newcommand{\todo}[1]{\textcolor{red}{#1}}
	
	
\title{Walk Around The World}
\author{Ryan Wells}
\date{\today}
\maketitle
\begin{abstract}
For many, cardiovascular exercise is an activity that we do not get
enough of\cite{exercise} . Studies have shown that physical exercise
between 1 hour a day for children over five years of age and as
little as 2.5 hours a week for adults has a huge positive impact on
our physical and emotional wellbeing\cite{govsurvey, amsurvey} .

This project aims to encourage users to do oudoor
cardiovascular exercise by positive encouragement through a Mobile
Application. By tracking the distance the user travels in each
exercise session through device GPS and translating this into
well-known outdoor pursuits allows the user to quantify their exercise
in terms of well known physical and cultural achievements. These
achievements will take badge-like form and other metrics of success
such as leaderboards will also exist. 

Outdoor pursuits will contain: actual routes such as the West Highland
Way and the climb to the top of Everest; popular culture references
such as ``Route 66'' and the (approximate) distance that \emph{Frodo}
and \emph{Sam} travelled in J.R.R.Tolkein's ``The Lord of the Rings'';
and Global distances such as the distance between capitol cities and
simple metrics such as the first one hundred miles. 

The user will exercise with a compatible device on their
person. During exercise the device will track and log the distance
that the user has travelled and add this to their accumulators. If
this device is their smart phone then the Mobile Application will
notify them immediately when they have completed an outdoor pursuit,
otherwise they will be notified when data is collected from the
dedicated hardware.

This application will encourage social exercise through outdoor
pursuits that can only be completed through teamwork with other users
that have met in real life. This intends to extend the influence of
rewarding exercise by encouraging users to include their peer group in
exercising with them and sharing their successes. For a user,
this is incentivised by unlocking achievements only achievable through
this social interaction.

An analysis of user trends such as: frequency and duration of exercise
with regards to duration of application use; change in exercise
duration when achievements are awarded during exercise; and quantity of
social interaction with regards to application use life cycle will be
used to indicate whether or not gamification techniques could have a
valid application in cardiovascular exercise. 

\end{abstract}
%\educationalconsent
\tableofcontents
%==============================================================================
\pagebreak
\pagenumbering{arabic}
\chapter{Introduction}\label{ch_intro}

\section{Motivation / Context}
The number of smart phones being sold per year is
increasing\cite{phones_gartner, phones_guardian} and nearly
three-quarters of exercisers use technology to support their
workouts in some way\cite{lifefitness}. This project aims to utilise
this mass market 
acceptance of technology in exercise and combine it with game like
elements to evaluate the effects of games and
exercise. Specifically, this project will look at outdoor
cardiovascular exercise where the user is required to exercise with
their phone with them.Studies have shown that physical exercise
between 1 hour a day for children over five years of age and as
little as 2.5 hours a week for adults has a huge positive impact on
our physical and emotional wellbeing\cite{govsurvey, amsurvey} .

The 2006 Health Survey for England\cite{exercise} states that: 
\begin{quote}
  Physical inactivity is associated with all-cause mortality and
  many chronic diseases, including ischaemic heart disease, diabetes,
  certain cancers, and obesity \dots \ Many people attribute their
  failure to achieve the target recommendations to a lack of time to
  take exercise. 
\end{quote} 
We intend to approach this problem by targeting the perception that
there is a ``lack of time to take exercise''. 

A later Health Survey for England in 2012 noted that the change in
definition as to what the exercise guidelines were implied that more
people than before were meeting the exercise guidelines
\cite{exercise_2012}. These guidelines allow for exercise sessions of
10 minutes or more where previously exercise only counted if it was
more than 30 minutes in length. Automatically this will include
more people than the previous survey as the criteria is less
restrictive, however we can use this to our advantage..

Since the guidelines now allow exercise sessions that can be as little
as 10 minutes we can use this to deal with the ``lack of time to take
exercise'' excuse and encourage users to use these small time
intervals to undertake exercise. 

We also want to break out of the monotony that exercise can
create. Running the same routes on a regular basis can become boring
and feel more like a chore than a reward. By creating a story that a
user can engage with we hope to challenge this problem. This project
allows the user to travel along routes that they are physically
nowhere near: even though you are running around Kelvin Park you could
be virtually running around Central Park in New York inside the app,
discovering photographs of the area as you run. 

The virtual routes that we provide do not need to be traditional
either. It could be enjoyable to say you have virtually travelled the
along the West Highland Way, but it would be far more satisfying to
say you have virtually travelled with \emph{Frodo} and \emph{Sam} on
their epic quest to save \emph{Middle Earth}, travelled the entire
circumference of the earth (\emph{walking around the world}), or by
finding a group of friends and collectively pooling your efforts have
\emph{walked to the moon and back}. By allowing the user to do what
would otherwise be impossible, we hope to encourage the user to want
to exercise. 

\section{Problem Statement}

The application of game like elements to exercise is nothing new. As
will be discussed in Section \ref{sec_comparison}, other applications
have taken different approaches to incentivising exercise. To the best
of my knowledge no other product uses exactly the same experience
as the one proposed in this dissertation.

The aim of this project is to create a gamified experience for outdoor
exercise, specifically walking, jogging and running. By creating a
gamified experience we aim to encourage and incentivise users to
exercise. 

\section{Targeted users}
\label{sec:targeted_users}
There are two major targeted user groups: the user who does not
exercise but does travel, and the user who exercises but is starting
to bore of the routine.

Consider an individual who has a relatively short distance to travel,
such as a commute to work,
and who doesn't exercise regularly. This individual could either
travel this distance on their own accord or use a service such as a
car or take public transport. With no external encouragement, and low
self control, the individual may be likely to take the car or public
transport and lose out on all benefits of exercise. 

By gamifying this travel time we hope that the user decides not to
take public transport or travel by car so that they can achieve the
goals of the game. This positive encouragement will help the user to
begin exercising and be rewarded for doing so. Gradually over time as
users start to engage with the game more often we hope that users
start to willingly extend their commute, colloquially ``taking the
long route'', or simply exercise more often just to ensure that they
receive an achievement. If this happens then the game has been
successful as through using this application the user has exercised
more than they would have without it. The short term rewards will
initially help these users the most: since they do not exercise
often they may need quicker encouragement to help push them to
exercise more regularly.

Also consider a user who frequently exercises and repeatedly travels
the same route, or the same collection of routes when exercising. This
individual may easily tire of the journey or bad weather may
discourage the individual just enough that they decide to move their
exercise to a controlled environment such as a gym where the benefits
are not as profound, or give up entirely. 

If we can encourage this user to continue exercising by giving the
user a reward structure they wish to invest in then we can help this
user continue to benefit from exercise. In this case the game is a
success if it diverts the user from changing away from their current
exercise regime. Long term rewards will benefit these users as they
can use these to help encourage themselves to maintain the current
rate that they are exercising at. These will be reinforced with the
rewards from the short term goals in a similar way to the previous
example. 

\section{Summary of Contributions}

This report presents the following contributions: 

\begin{itemize}
  \item A brief overview of gamification;
  \item An analysis of the current state of the market for exercise
    apps and devices;
  \item An exploration of the requirements, design decisions and
    implementation of ``Urban Explorer'' - the application built for
    the analysis of this type of gamification;
  \item The ``Urban Explorer'' application, which can also be found on
    the Google Play Store \cite{app_store_link};
  \item The results from user evaluation and testing.
\end{itemize}

\section{Outline of the Report}

We will start by examining the current state of the market in Chapter
\ref{ch_background} by
discussing other implementations of gamification in mobile
apps. Through this examination, we will discover the benefits and
pitfalls of each implementation and use this to form the basis of our
implementation. 

We define the overall goals of the project in Chapter \ref{ch_method},
specifically how we will use gamification to deliver our
implementation and what we hope to achieve in the scope of this
project.

An analysis of our implementation will then be presented in Chapter
\ref{ch_results}. This is based on user testing and experience, and evaluation of the
overall suitability of the platform.

\chapter{Related Work and Background}\label{ch_background}

\section{Gamification and technology in exercise}

Life Fitness in 2012 released a survey concerning technology in the
exercise environment\cite{lifefitness} and it was found that over half of
gym users used a smartphone or tablet when exercising. This study
focused solely on gym users which is not our target market for this
product (we are targeting users who exercise outdoors), however it is
not unreasonable to assume similar statistics for our target
demographic. We use this metric as a basis of platform validity: we
know people use this platform in our targeted situation and so we are
justified to capitalise on this.

\subsection{Defining gamification in our context}
Sridharan, Hrishikesh, \& Raj\cite{academic_gamification}


\todo{This may also be relevant, need to get when in uni
  \url{http://dl.acm.org/citation.cfm?id=1125805}}



\section{Comparison of related applications}\label{sec_comparison}
Incentivising exercise is not a new concept in the mobile market. From
simple applications that mimic a pedometer to immersive alternate
realities, there has been a varied approach to encouraging
exercise. These approaches can be shown through the following
applications.

\subsection{Charity Miles}
Charity Miles encourages you to exercise by facilitating you to raise
awareness of charities and help
others as you exercise. For every mile that you run or walk \$0.25 is
donated to a charity of your choice, and for every mile you cycle \$0.10
is donated. The money currently comes from a pool of \$1,000,000
gathered by the parent company - so you as an end user don't have to pay
anything. 

The money you accrue during an exersize session is called your
``sponsorship'' and is only passed onto your charity when you share
your success to a social network (facebook or
twitter)\cite{charitymiles_terms}. Charity Miles say this is to
increase awareness of the charity, but it is also a good advertising
tool for them.

The incentive here is clear - run to donate. In essence, you are both
being paid to exercise and exercising to help out a charity. The
monetary backing allows you to quantify your exercise in terms you are
very familiar with whilst generating a feeling of satisfaction by
helping a charity.

The main drawback here is that there is no defined behaviour for what
happens when the \$1,000,000 pool runs out. From a business
perspective, the app could easily be re-purposed for non-centralised
pool funding where the end user could find local sponsorship. However
this entirely changes the user experience and explicitly requires
effort on the users side to gather this sponsorship. 

\subsection{Zombies, Run!}
Zombies, Run! is a companion app that encourages you to exercise through
audio cues, rich story and a base-building game. As you run you are
being ``chased'' by a hoard of zombies that get closer and further
away from you as you exercise - the closer they are to catching you
the louder they become. 

In the app you have a base from where you defend yourself from the
zombie hoard and are required to leave the base to gather
supplies. The story narritive tells you what you need to gather and
you find these supplies based on the distance you travel and the route
you take. Your success is measured by the fortifications of your base,
which you can improve by running faster/further than before.

You are required to wear headphones and carry your mobile phone while
exercising. The headphones give you audio queues to immerse you in the
story, but it is a requirement of the game experience that you listen
to the soundscape provided - you cannot listen to your own music. As
endearing as groaning zombies may be, it is understandable that users
may tire of hearing it. Also, a user may be used to listening to their
own music and use that to zone out of the exercise so some users might
be reluctant to adopt this.


\subsection{WalkJogRun}
WalkJogRun helps you discover new routes that are geographically
close to where you are and record your favourite routes to share with
your friends. From your current location it finds and
proposes new exercise routes given a certain route length. It can
provide turn-by-turn direction help when you are exercising to ensure
you do not get lost and will also record an exact route as you run so
you can create a new route with the minimum of effort. 

Your workout is saved and can be later analysed so you can accurately
monitor your progress.

Since the routes are user provided, there is no guarantee that the
routes are safely traversable or applicable to all types of exercise -
a route that can be walked may not be applicable for a runner due to
terrain type, for example. 

There is no clear gamified experience in this application, so the
incentive is self-provided. The user can only use this to improve
their exercise experience and there is not much in the way of encouragement.

\subsection{Fitbit}
I have included a comparison with Fitbit products to display other
mediums in which this application can be used. Although it is not
mainly based around a mobile phone use it is an interesting comparison.

Fitbit provide a collection of dedicated hardware devices capable of
accurately measuring your distance travelled (and in some models the
height you have ascended in terms of ``flights of stairs'') as a means
of tracking your activity. At the core, it is a glorified pedometer
with web and phone apps to accompany it and display progress. The
devices and apps also make a prediction about the calories you burn each
day based upon the distance you travel when wearing the device, your
weight and stride length. 

Each device can show you the progress toward your daily goal (either
steps taken or predicted calorie burn) with various visualisation
techniques - the simplest of which is a row of five LED lights that
fill up as you near your goal. Synchronization with compatible phones,
or through a computer if your device is not compatible, allows a more
in depth visualization of your progress.

The gamification aspect is inherent in the design of these products -
you set your own goal and you can easily view your progress at a
glance. Upon completion of your goal the device vibrates and lights up
to notify you. You are also awarded badges for daily completion such
as walking 10,000 steps a day and lifetime completion goals such as
travelling 100 miles in total.

There is a clear downside to this - you are required to purchase a
bespoke device to fully utilize the experience provided by
Fitbit. When compared to an app that may be \$1.99 or thereabouts it is
a much larger startup cost than the previous examples. However this
larger startup cost may be an encouragement unto itself: the user may
feel obligated to commit to using the device since they are more
financially invested, which may help with the onboarding required to
fully benefit from these systems.

Most of these devices are reasonably discreet, the more advanced the
devices get the larger they become. Since these devices must be worn
mainly on the wrist there are social implications involved. Does the
user want everyone to know they are using a fitness tracker? The other
applications are more discreet by design, since they are a mobile app
they can be hidden easier. 

\subsection{General Comments}
Each of the above products utilise gamification in an effective way
and each implementation differs in how they utilise game elements. 
\begin{table}[h]
  \centering
  \begin{tabular}{ | c | c | c | c |} \hline
    Name & Interface & Requirements & User base  \\ \hline
    Charity Miles & Mobile App & App and GPS & 10K - 50K\\ \hline
    Zombies, Run! & Mobile App & App and GPS, headphones & 1M - 5M\\ \hline
    WalkJogRun & Mobile App & App and GPS  & unknown \\ \hline
    Fitbit & Dedicated hardware and Mobile App & Dedicated Hardware & 1M - 5M\\ \hline
  \end{tabular}
  \caption{Comparison of exercise gamification implementations from
    the Google Play store}
\end{table}

%\todo{Summarising a paper: (1) Context, (2) Problem, (3) Solution, (4) Evaluation, (5) Impact.}


%\todo{Places to go: ACM Portal, Google Scholar, Citeseer}

%\todo{Journals: IPM, JIR, ACM TOIS}
%\todo{Conferences: ACM SIGIR, ACM CHI, ACM CIKM, European Conference in IR (ECIR), Information Interaction in Context (IIiX), etc.}


%\todo{Types or Styles of Papers: Theoretical, Empirical, Conceptual, Applications Based.}


%\todo{It is your job to add value and show how the background work relates to your project}

%\begin{itemize}
%	\item What is it about?
%	\item Why would i read it? What is of value in it?
%	\item What are the main contributions in the paper?
%	\item What are the main issues in the paper?
%	\item What are the advantages and disadvantages of the approach/solution proposed?
%	\item What are the limitations of the work?
%	\item What does the paper claims does the paper make? And are they supported?
%	\item What do other people think of the paper? Who has caned it?
%	\item Consider whether the paper is seminal or delta?
%	\item What did you learn from this paper?
%	\item Who else has done work in this area?
%	\item How does this work stand out?
%	\item How does it relate to the research questions?
%\end{itemize}









\chapter{Methodology}\label{ch_method}


\section{Walkthrough of Wireframes}
When the app is launched it silently registers with the server
allowing the user to use the app immediately. The user is then shown
the middle-top screen.
\begin{enumerate}
  \item From the middle top screen, the user can follow arrow 1 by
    clicking on the middle button ``Pick Mission'' to pick a Mission
    (Run around Arran or Egg for example) and then pick a start and
    end location. After confirming these choices, the user is taken
    back to the middle top screen, or at any time can click the
    ``Home'' button to return. 
  \item The user can also view their current acheivements by clicking
    the ``Achievements'' button on the middle top screen, following
    arrow 2. These achievements will be grouped by tabs by category -
    Distance, Time, Stage and Mission based achievements.
  \item The user can follow arrow 3 from the middle top screen to
    notify the app that they are starting an exercise period, telling
    the app to track their distance. If a Mission and start and end
    location are not picked (as in point 1) then they will instead be
    redirected to this screen and are unable to start exercising until
    this choice has been made. Once they have successfully advanced to
    this screen, it will display their current progress as they move
    showing the user how close to completion of their current stage
    and overall route they are. 
  \item When the user has finished exercising, they will click the
    ``End Session'' button and be taken to the first summary screen -
    following arrow 4. Here statistics from their exercise will be
    shown and the option to share this on several social media
    outlets.
  \item The user can then move to the second and final summary screen,
    following arrow 5, where they will be shown any achievements they
    were awarded during that session. The user will also have the
    option to share these on social media outlets. From here, the user
    can click the ``Home'' button and be taken back to the middle top
    screen. 
\end{enumerate}

Latency of gps might mean that a user drops out of connection for
finding their location. We should make sure that the game makes
``reasonable'' accommodations about this - first iteration will save
the game object after a certain period of time if it isn't formally
closed, other iterations might try and build on this.


\chapter{Results and Analysis}\label{ch_results}
Here are some results

\chapter{Conclusion}\label{ch_conclusion}
Based on the results from the evaluation and the general feedback from
users, this could be a credible application of gamification. Users
liked the way encouragement was provided and the few users that did
engage completely with the app successfully received the
rewards. Further study would show if this is true in general but it
would certainly show what additions would be worthwhile to improve the
experience. 

Overall, the author is satisfied with the results of the project. The
skills learned by working with the technologies are rewarding in their
own right but it has made the author consider the way that he views
exercise. There has been a conscious decision to stop using public
transport in favour of gaining the health benefits of outdoor
exercise, and a new outlook on gamification and games has been
discovered. 

There is more to gamification than just an application of the
standard techniques. The story that can be created to
surround this interaction is infinitely more engaging than core
concepts in isolation. Executing a task just to receive a badge has
the potential to be satisfying but this satisfaction may wear off
quickly. We generally apply gamification to processes that users
either loathe to do or those processes that users simply do not do due
to their current behaviour. If we only encourage the user to do these
processes so they achieve a badge then after time the user will come
to realise that the process is no different to what it was before
except for the addition of a reward at the end. These rewards could
become meaningless to the user if achieving a new reward doesn't
improve their status or get any encouragement towards an even greater
goal. Each element of a game is critically important to the overall
success of the game and more emphasis on the story in particular for
this project may have created a more engaged user base.

It is the belief of the author that the future of influencing users
will be through gamification. We were taught by games throughout our
childhood, from learning songs that teach us the alphabet, playing
games to learn hand-eye coordination, and playing board games to learn
strategy and skill, so there is little reason to suggests this
should stop being beneficial just because we grow up. Growing up
doesn't mean that games stop helping us, we just expect the reward to
be more physical and well defined. We want to be challenged in a way
that we deem to be more appropriate to our perceived skill level, and
moreover the rewards provided by the games we played when we were
younger are no longer appealing to us. We yearn for the satisfaction
that games gave us and if we cannot find a game that fulfils this
satisfaction then often the thought of gaming can be written off as a
frivolous activity that is not worth our time. Gamification can help
rediscover this joy, or help us continue our enjoyment, but there is a
larger issue that should also be tackled.

It is the authors view that there is a stigma around gaming,
especially in a digital form, that it is a sub-culture that you don't
want to be explicitly associated with. There is a reluctance from some
to identify themselves as a ``gamer'' for fear of being related to
their grossly inaccurate stereotype they have of ``gamers''. Every one
of us at some point in our lives has ``gamed'' in some form: whether
we've played sports, a tabletop game with our friends or an online
role-playing game with millions of users across the world. By allowing
a larger, more open acceptance of digital gaming we will, as a
community, allow the benefits of gamification to reach a wider
audience, and maybe rekindle the love of games we had when we were
younger. 


\section{Acknowledgements}
I would like to thank ...




\bibliographystyle{unsrt}
\bibliography{wells-2013-urbanexplorer}
\end{document}
