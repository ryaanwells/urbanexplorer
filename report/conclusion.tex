\chapter{Conclusion}\label{ch_conclusion}
Based on the results from the evaluation and the general feedback from
users, this could be a credible application of gamification. Users
liked the way encouragement was provided and the few users that did
engage completely with the app successfully received the
rewards. Further study would show if this is true in general but it
would certainly show what additions would be worthwhile to improve the
experience. 

Overall, the author is satisfied with the results of the project. The
skills learned by working with the technologies are rewarding in their
own right but it has made the author consider the way that he views
exercise. There has been a conscious decision to stop using public
transport in favour of gaining the health benefits of outdoor
exercise, and a new outlook on gamification and games has been
discovered. 

There is more to gamification than just an application of the
standard techniques. The story that can be created to
surround this interaction is infinitely more engaging than core
concepts in isolation. Executing a task just to receive a badge has
the potential to be satisfying but this satisfaction may wear off
quickly. We generally apply gamification to processes that users
either loathe to do or those processes that users simply do not do due
to their current behaviour. If we only encourage the user to do these
processes so they achieve a badge then after time the user will come
to realise that the process is no different to what it was before
except for the addition of a reward at the end. These rewards could
become meaningless to the user if achieving a new reward doesn't
improve their status or get any encouragement towards an even greater
goal. Each element of a game is critically important to the overall
success of the game and more emphasis on the story in particular for
this project may have created a more engaged user base.

It is the belief of the author that the future of influencing users
will be through gamification. We were taught by games throughout our
childhood, from learning songs that teach us the alphabet, playing
games to learn hand-eye coordination, and playing board games to learn
strategy and skill, so there is little reason to suggests this
should stop being beneficial just because we grow up. Growing up
doesn't mean that games stop helping us, we just expect the reward to
be more physical and well defined. We want to be challenged in a way
that we deem to be more appropriate to our perceived skill level, and
moreover the rewards provided by the games we played when we were
younger are no longer appealing to us. We yearn for the satisfaction
that games gave us and if we cannot find a game that fulfils this
satisfaction then often the thought of gaming can be written off as a
frivolous activity that is not worth our time. Gamification can help
rediscover this joy, or help us continue our enjoyment, but there is a
larger issue that should also be tackled.

It is the authors view that there is a stigma around gaming,
especially in a digital form, that it is a sub-culture that you don't
want to be explicitly associated with. There is a reluctance from some
to identify themselves as a ``gamer'' for fear of being related to
their grossly inaccurate stereotype they have of ``gamers''. Every one
of us at some point in our lives has ``gamed'' in some form: whether
we've played sports, a tabletop game with our friends or an online
role-playing game with millions of users across the world. By allowing
a larger, more open acceptance of digital gaming we will, as a
community, allow the benefits of gamification to reach a wider
audience, and maybe rekindle the love of games we had when we were
younger. 
