\chapter{Methodology}\label{ch_method}

\section{Walkthrough of Wireframes}
When the app is launched it silently registers with the server
allowing the user to use the app immediately. The user is then shown
the middle-top screen.
\begin{enumerate}
  \item From the middle top screen, the user can follow arrow 1 by
    clicking on the middle button ``Pick Mission'' to pick a Mission
    (Run around Arran or Egg for example) and then pick a start and
    end location. After confirming these choices, the user is taken
    back to the middle top screen, or at any time can click the
    ``Home'' button to return. 
  \item The user can also view their current acheivements by clicking
    the ``Achievements'' button on the middle top screen, following
    arrow 2. These achievements will be grouped by tabs by category -
    Distance, Time, Stage and Mission based achievements.
  \item The user can follow arrow 3 from the middle top screen to
    notify the app that they are starting an exercise period, telling
    the app to track their distance. If a Mission and start and end
    location are not picked (as in point 1) then they will instead be
    redirected to this screen and are unable to start exercising until
    this choice has been made. Once they have successfully advanced to
    this screen, it will display their current progress as they move
    showing the user how close to completion of their current stage
    and overall route they are. 
  \item When the user has finished exercising, they will click the
    ``End Session'' button and be taken to the first summary screen -
    following arrow 4. Here statistics from their exercise will be
    shown and the option to share this on several social media
    outlets.
  \item The user can then move to the second and final summary screen,
    following arrow 5, where they will be shown any achievements they
    were awarded during that session. The user will also have the
    option to share these on social media outlets. From here, the user
    can click the ``Home'' button and be taken back to the middle top
    screen. 
\end{enumerate}
