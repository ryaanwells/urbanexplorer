\chapter{Design and Specification}\label{ch_method}

\section{Aim/vision}

The idea of the game is to walk around the world. This goal cannot be achieved overnight so we break it up into routes such as Glasgow to Edinburgh and each route is broken up into manageable stages of a few kilometres each. Users are awarded badges based on stages/routes completed and overall distance and time. By doing this we allow the user to feel a sense of achievement as they are working towards a larger goal without making the user change anything about their exercise. The app allows a user to travel around parts of the world without physically having to be there, inviting them to realise the scale of the world whilst gaining the positives benefits of outdoor exercise. 

This approach is to test whether or not this type of encouragement works with outdoor exercise. Metrics will be tracked for individual users to monitor their exercise duration and frequency and 
Encourage people to exercise
Provide incentives that become intrinsically rewarding.  
Provide a platform to achieve this.
Provide long and short term goals for a sense of achievement. 
Transform an abstract concept of distance and time into a physical goal that a user can get. This is used to facilitate the previous points.

Why do we need this application? High level vision - gist.
??



\section{Specification}
\subsection{Platform}
The mobile application is developed using PhoneGap which utilises
javascript, css and html to package a web app as a mobile app. The
decision to use PhoneGap instead of building a native app is so we can
deploy to multiple platforms easily without needing to change the code
base and because I come from a web app background which is ideal for
PhoneGap.

The web application is written in Python and uses Django middleware
for interfacing with the database and creation of specific workflow
interactions, alongside Tastypie for managing and building a REST
style API.

The server will not hold state outside the supporting database and
will be used solely as a REST API. This is a decision to keep the
sever implementation clean and simple. There is also no real gain to
maintaining state as the underlying database objects all ready hold
the required information.

Implementation/Design specification
How does the game run
What are the metrics of success
MoSCoW


\section{Walkthrough of Wireframes}
When the app is launched it silently registers with the server
allowing the user to use the app immediately. The user is then shown
the middle-top screen.
\begin{enumerate}
  \item From the middle top screen, the user can follow arrow 1 by
    clicking on the middle button ``Pick Mission'' to pick a Mission
    (Run around Arran or Egg for example) and then pick a start and
    end location. After confirming these choices, the user is taken
    back to the middle top screen, or at any time can click the
    ``Home'' button to return. 
  \item The user can also view their current acheivements by clicking
    the ``Achievements'' button on the middle top screen, following
    arrow 2. These achievements will be grouped by tabs by category -
    Distance, Time, Stage and Mission based achievements.
  \item The user can follow arrow 3 from the middle top screen to
    notify the app that they are starting an exercise period, telling
    the app to track their distance. If a Mission and start and end
    location are not picked (as in point 1) then they will instead be
    redirected to this screen and are unable to start exercising until
    this choice has been made. Once they have successfully advanced to
    this screen, it will display their current progress as they move
    showing the user how close to completion of their current stage
    and overall route they are. 
  \item When the user has finished exercising, they will click the
    ``End Session'' button and be taken to the first summary screen -
    following arrow 4. Here statistics from their exercise will be
    shown and the option to share this on several social media
    outlets.
  \item The user can then move to the second and final summary screen,
    following arrow 5, where they will be shown any achievements they
    were awarded during that session. The user will also have the
    option to share these on social media outlets. From here, the user
    can click the ``Home'' button and be taken back to the middle top
    screen. 
\end{enumerate}

Latency of gps might mean that a user drops out of connection for
finding their location. We should make sure that the game makes
``reasonable'' accommodations about this - first iteration will save
the game object after a certain period of time if it isn't formally
closed, other iterations might try and build on this.

