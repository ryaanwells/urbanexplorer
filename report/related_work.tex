\chapter{Related Work or Background}\label{ch_background}

\section{Comparison of related applications}
Incentivising exercise is not a new concept in the mobile market. From
simple applications that mimic a pedometer to immersive alternate
realities, there has been a varied approach to encouraging
exercise. These approaches can be shown through the following
applications.

\todo{Add in opinion from using the app}

\subsection{Charity Miles}
Charity Miles encourages you to exercise by facilitating you to help
others as you exercise. For every mile that you run \$0.25 is
donated to a charity of your choice, and for every mile you cycle \$0.10
is donated. The money currently comes from a pool gathered by the
parent company so you as an end user don't have to pay anything.

The incentive here is clear - run to donate. In essence, you are both
being paid to exercise and exercising to help out a charity. The
monetary backing allows you to quantify your exercise in terms you are
very familiar with whilst generating the feeling of satisfaction about
helping a charity.

\subsection{Run Zombie Run}
Zombie Run! is a companion app that encourages you to exercise through audio cues. As you run you are being ``chased'' by a hoard of zombies that get closer and further away from you as you exercise - the closer they are to catching you the louder they become. The game measures your running distance and this is used as a measure of your success. 


\subsection{WalkJogRun}
Out of the other apps compared this far, WalkJogRun is the most similar to UrbanExplorer. WalkJogRun is also centred around routes but unlike UrbanExplorer is concerned with routes that are geographically close to where you are. From your current location it finds and proposes new exercise routes given a certain route length and helps to match you with other people in your area to help create running groups.

This app uses the social aspect of exercising alongside facilitating your discovery of new routes to incentivise exercise.

There may be limitations to the app's reuse if the routes provided are either not traversable or are consistently the same. The social aspect is a very good approach but also requires a user base to be useful to any end user.

\subsection{Nike}

\subsection{Fitbit}

\subsection{Garmin}


\begin{tabular}{ | c | c | c | c  |} \hline
 Name & Hardware & Requirements \\ \hline
 Charity Miles & Mobile (iOS/android) & GPS \\ \hline
 Zombies, Run! & Mobile (iOS/android) & GPS, headphones \\ \hline
 WalkJogRun & Mobile (iOS/android) & GPS \\ \hline
 Fitbit & Dedicated hardware and Mobile (iOS/android) & \\ \hline
 Nike & Dedicated hardware and Mobile (iOS/android) & \\ \hline
 Garmin & Dedicated hardware and webapp & GPS \\ \hline 
\end{tabular}
%\todo{Summarising a paper: (1) Context, (2) Problem, (3) Solution, (4) Evaluation, (5) Impact.}


%\todo{Places to go: ACM Portal, Google Scholar, Citeseer}

%\todo{Journals: IPM, JIR, ACM TOIS}
%\todo{Conferences: ACM SIGIR, ACM CHI, ACM CIKM, European Conference in IR (ECIR), Information Interaction in Context (IIiX), etc.}


%\todo{Types or Styles of Papers: Theoretical, Empirical, Conceptual, Applications Based.}


%\todo{It is your job to add value and show how the background work relates to your project}

%\begin{itemize}
%	\item What is it about?
%	\item Why would i read it? What is of value in it?
%	\item What are the main contributions in the paper?
%	\item What are the main issues in the paper?
%	\item What are the advantages and disadvantages of the approach/solution proposed?
%	\item What are the limitations of the work?
%	\item What does the paper claims does the paper make? And are they supported?
%	\item What do other people think of the paper? Who has caned it?
%	\item Consider whether the paper is seminal or delta?
%	\item What did you learn from this paper?
%	\item Who else has done work in this area?
%	\item How does this work stand out?
%	\item How does it relate to the research questions?
%\end{itemize}








