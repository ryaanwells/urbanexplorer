\chapter{Related Work and Background}\label{ch_background}

\section{Gamification and technology in exercise}

Life Fitness in 2012 released a survey concerning technology in the
exercise environment\cite{lifefitness} and it was found that over half of
gym users used a smartphone or tablet when exercising. This study
focused solely on gym users which is not our target market for this
product (we are targeting users who exercise outdoors), however it is
not unreasonable to assume similar statistics for our target
demographic. We use this metric as a basis of platform validity: we
know people use this platform in our targeted situation and so we are
justified to capitalise on this.

\subsection{Defining gamification in our context}
Sridharan, Hrishikesh, \& Raj\cite{academic_gamification}


\todo{This may also be relevant, need to get when in uni
  \url{http://dl.acm.org/citation.cfm?id=1125805}}



\section{Comparison of related applications}\label{sec_comparison}
Incentivising exercise is not a new concept in the mobile market. From
simple applications that mimic a pedometer to immersive alternate
realities, there has been a varied approach to encouraging
exercise. These approaches can be shown through the following
applications.

\subsection{Charity Miles}
Charity Miles encourages you to exercise by facilitating you to raise
awareness of charities and help
others as you exercise. For every mile that you run or walk \$0.25 is
donated to a charity of your choice, and for every mile you cycle \$0.10
is donated. The money currently comes from a pool of \$1,000,000
gathered by the parent company - so you as an end user don't have to pay
anything. 

The money you accrue during an exersize session is called your
``sponsorship'' and is only passed onto your charity when you share
your success to a social network (facebook or
twitter)\cite{charitymiles_terms}. Charity Miles say this is to
increase awareness of the charity, but it is also a good advertising
tool for them.

The incentive here is clear - run to donate. In essence, you are both
being paid to exercise and exercising to help out a charity. The
monetary backing allows you to quantify your exercise in terms you are
very familiar with whilst generating a feeling of satisfaction by
helping a charity.

The main drawback here is that there is no defined behaviour for what
happens when the \$1,000,000 pool runs out. From a business
perspective, the app could easily be re-purposed for non-centralised
pool funding where the end user could find local sponsorship. However
this entirely changes the user experience and explicitly requires
effort on the users side to gather this sponsorship. 

\subsection{Zombies, Run!}
Zombies, Run! is a companion app that encourages you to exercise through
audio cues, rich story and a base-building game. As you run you are
being ``chased'' by a hoard of zombies that get closer and further
away from you as you exercise - the closer they are to catching you
the louder they become. 

In the app you have a base from where you defend yourself from the
zombie hoard and are required to leave the base to gather
supplies. The story narritive tells you what you need to gather and
you find these supplies based on the distance you travel and the route
you take. Your success is measured by the fortifications of your base,
which you can improve by running faster/further than before.

You are required to wear headphones and carry your mobile phone while
exercising. The headphones give you audio queues to immerse you in the
story, but it is a requirement of the game experience that you listen
to the soundscape provided - you cannot listen to your own music. As
endearing as groaning zombies may be, it is understandable that users
may tire of hearing it. Also, a user may be used to listening to their
own music and use that to zone out of the exercise so some users might
be reluctant to adopt this.


\subsection{WalkJogRun}
WalkJogRun helps you discover new routes that are geographically
close to where you are and record your favourite routes to share with
your friends. From your current location it finds and
proposes new exercise routes given a certain route length. It can
provide turn-by-turn direction help when you are exercising to ensure
you do not get lost and will also record an exact route as you run so
you can create a new route with the minimum of effort. 

Your workout is saved and can be later analysed so you can accurately
monitor your progress.

Since the routes are user provided, there is no guarantee that the
routes are safely traversable or applicable to all types of exercise -
a route that can be walked may not be applicable for a runner due to
terrain type, for example. 

There is no clear gamified experience in this application, so the
incentive is self-provided. The user can only use this to improve
their exercise experience and there is not much in the way of encouragement.

\subsection{Fitbit}
I have included a comparison with Fitbit products to display other
mediums in which this application can be used. Although it is not
mainly based around a mobile phone use it is an interesting comparison.

Fitbit provide a collection of dedicated hardware devices capable of
accurately measuring your distance travelled (and in some models the
height you have ascended in terms of ``flights of stairs'') as a means
of tracking your activity. At the core, it is a glorified pedometer
with web and phone apps to accompany it and display progress. The
devices and apps also make a prediction about the calories you burn each
day based upon the distance you travel when wearing the device, your
weight and stride length. 

Each device can show you the progress toward your daily goal (either
steps taken or predicted calorie burn) with various visualisation
techniques - the simplest of which is a row of five LED lights that
fill up as you near your goal. Synchronization with compatible phones,
or through a computer if your device is not compatible, allows a more
in depth visualization of your progress.

The gamification aspect is inherent in the design of these products -
you set your own goal and you can easily view your progress at a
glance. Upon completion of your goal the device vibrates and lights up
to notify you. You are also awarded badges for daily completion such
as walking 10,000 steps a day and lifetime completion goals such as
travelling 100 miles in total.

There is a clear downside to this - you are required to purchase a
bespoke device to fully utilize the experience provided by
Fitbit. When compared to an app that may be \$1.99 or thereabouts it is
a much larger startup cost than the previous examples. However this
larger startup cost may be an encouragement unto itself: the user may
feel obligated to commit to using the device since they are more
financially invested, which may help with the onboarding required to
fully benefit from these systems.

Most of these devices are reasonably discreet, the more advanced the
devices get the larger they become. Since these devices must be worn
mainly on the wrist there are social implications involved. Does the
user want everyone to know they are using a fitness tracker? The other
applications are more discreet by design, since they are a mobile app
they can be hidden easier. 

\subsection{General Comments}
Each of the above products utilise gamification in an effective way
and each implementation differs in how they utilise game elements. 
\begin{table}[h]
  \centering
  \begin{tabular}{ | c | c | c | c |} \hline
    Name & Interface & Requirements & User base  \\ \hline
    Charity Miles & Mobile App & App and GPS & 10K - 50K\\ \hline
    Zombies, Run! & Mobile App & App and GPS, headphones & 1M - 5M\\ \hline
    WalkJogRun & Mobile App & App and GPS  & unknown \\ \hline
    Fitbit & Dedicated hardware and Mobile App & Dedicated Hardware & 1M - 5M\\ \hline
  \end{tabular}
  \caption{Comparison of exercise gamification implementations from
    the Google Play store}
\end{table}

%\todo{Summarising a paper: (1) Context, (2) Problem, (3) Solution, (4) Evaluation, (5) Impact.}


%\todo{Places to go: ACM Portal, Google Scholar, Citeseer}

%\todo{Journals: IPM, JIR, ACM TOIS}
%\todo{Conferences: ACM SIGIR, ACM CHI, ACM CIKM, European Conference in IR (ECIR), Information Interaction in Context (IIiX), etc.}


%\todo{Types or Styles of Papers: Theoretical, Empirical, Conceptual, Applications Based.}


%\todo{It is your job to add value and show how the background work relates to your project}

%\begin{itemize}
%	\item What is it about?
%	\item Why would i read it? What is of value in it?
%	\item What are the main contributions in the paper?
%	\item What are the main issues in the paper?
%	\item What are the advantages and disadvantages of the approach/solution proposed?
%	\item What are the limitations of the work?
%	\item What does the paper claims does the paper make? And are they supported?
%	\item What do other people think of the paper? Who has caned it?
%	\item Consider whether the paper is seminal or delta?
%	\item What did you learn from this paper?
%	\item Who else has done work in this area?
%	\item How does this work stand out?
%	\item How does it relate to the research questions?
%\end{itemize}








